% Options for packages loaded elsewhere
\PassOptionsToPackage{unicode}{hyperref}
\PassOptionsToPackage{hyphens}{url}
%
\documentclass[
  letterpaper,
  ignorenonframetext,
  aspectratio=43,
  handout,
  12pt]{beamer}
\usepackage{pgfpages}
\setbeamertemplate{caption}[numbered]
\setbeamertemplate{caption label separator}{: }
\setbeamercolor{caption name}{fg=normal text.fg}
\beamertemplatenavigationsymbolsempty
% Prevent slide breaks in the middle of a paragraph
\widowpenalties 1 10000
\raggedbottom
\setbeamertemplate{part page}{
  \centering
  \begin{beamercolorbox}[sep=16pt,center]{part title}
    \usebeamerfont{part title}\insertpart\par
  \end{beamercolorbox}
}
\setbeamertemplate{section page}{
  \centering
  \begin{beamercolorbox}[sep=12pt,center]{part title}
    \usebeamerfont{section title}\insertsection\par
  \end{beamercolorbox}
}
\setbeamertemplate{subsection page}{
  \centering
  \begin{beamercolorbox}[sep=8pt,center]{part title}
    \usebeamerfont{subsection title}\insertsubsection\par
  \end{beamercolorbox}
}
\AtBeginPart{
  \frame{\partpage}
}
\AtBeginSection{
  \ifbibliography
  \else
    \frame{\sectionpage}
  \fi
}
\AtBeginSubsection{
  \frame{\subsectionpage}
}
\usepackage{amsmath,amssymb}
\usepackage{lmodern}
\usepackage{iftex}
\ifPDFTeX
  \usepackage[T1]{fontenc}
  \usepackage[utf8]{inputenc}
  \usepackage{textcomp} % provide euro and other symbols
\else % if luatex or xetex
  \usepackage{unicode-math}
  \defaultfontfeatures{Scale=MatchLowercase}
  \defaultfontfeatures[\rmfamily]{Ligatures=TeX,Scale=1}
\fi
\usetheme[]{metropolis}
% Use upquote if available, for straight quotes in verbatim environments
\IfFileExists{upquote.sty}{\usepackage{upquote}}{}
\IfFileExists{microtype.sty}{% use microtype if available
  \usepackage[]{microtype}
  \UseMicrotypeSet[protrusion]{basicmath} % disable protrusion for tt fonts
}{}
\makeatletter
\@ifundefined{KOMAClassName}{% if non-KOMA class
  \IfFileExists{parskip.sty}{%
    \usepackage{parskip}
  }{% else
    \setlength{\parindent}{0pt}
    \setlength{\parskip}{6pt plus 2pt minus 1pt}}
}{% if KOMA class
  \KOMAoptions{parskip=half}}
\makeatother
\usepackage{xcolor}
\IfFileExists{xurl.sty}{\usepackage{xurl}}{} % add URL line breaks if available
\IfFileExists{bookmark.sty}{\usepackage{bookmark}}{\usepackage{hyperref}}
\hypersetup{
  hidelinks,
  pdfcreator={LaTeX via pandoc}}
\urlstyle{same} % disable monospaced font for URLs
\newif\ifbibliography
% Make links footnotes instead of hotlinks:
\DeclareRobustCommand{\href}[2]{#2\footnote{\url{#1}}}
\setlength{\emergencystretch}{3em} % prevent overfull lines
\providecommand{\tightlist}{%
  \setlength{\itemsep}{0pt}\setlength{\parskip}{0pt}}
\setcounter{secnumdepth}{-\maxdimen} % remove section numbering
\usepackage{pgfpages}
\pgfpagesuselayout{2 on 1}
\providecommand{\tightlist}{%
\setlength{\itemsep}{0pt}\setlength{\parskip}{0pt}}
\makeatletter
\makeatother
\let\Oldincludegraphics\includegraphics
\renewcommand{\includegraphics}[2][]{\Oldincludegraphics[width=\textwidth,height=0.7\textheight,keepaspectratio]{#2}}
\ifLuaTeX
  \usepackage{selnolig}  % disable illegal ligatures
\fi

\author{}
\date{}

\begin{document}

\begin{frame}{AE 737: Mechanics of Damage Tolerance}
\protect\hypertarget{ae-737-mechanics-of-damage-tolerance}{}
Lecture 18 - The Boeing Method

Dr.~Nicholas Smith

Wichita State University, Department of Aerospace Engineering

April 5, 2021
\end{frame}

\begin{frame}{schedule}
\protect\hypertarget{schedule}{}
\begin{itemize}
\tightlist
\item
  5 Apr - Boeing Method
\item
  7 Apr - Cycle counting
\item
  9 Apr - Homework 6 Self-grade, Homework 7 Due
\item
  12 Apr - Crack retardation
\item
  14 Apr - Finite Elements in Fracture
\item
  16 Apr - Homework 7 Self-grade, Homework 8 Due
\item
  19 Apr - Exam Review
\item
  21 Apr - Exam 2
\end{itemize}
\end{frame}

\begin{frame}{outline}
\protect\hypertarget{outline}{}
\begin{itemize}
\tightlist
\item
  boeing method
\end{itemize}
\end{frame}

\hypertarget{boeing-method}{%
\section{boeing method}\label{boeing-method}}

\begin{frame}{boeing method}
\protect\hypertarget{boeing-method-1}{}
\begin{itemize}
\tightlist
\item
  Whether integrating numerically or analytically, it is time-consuming
  to consider multiple repeated loads
\item
  It is particularly difficult to consider flight loads, which can vary
  by ``mission''
\item
  For example, an aircraft may fly three different routes, in no
  particular order, but with a known percentage of time spent in each
  route
\item
  Traditional methods would use a random mix of each load spectra
\end{itemize}
\end{frame}

\begin{frame}{boeing method}
\protect\hypertarget{boeing-method-2}{}
\begin{itemize}
\tightlist
\item
  The Boeing Method combines each repeatable load spectrum into one
  single equivalent cycle
\item
  Note: this is ch.~20 in the text
\end{itemize}
\end{frame}

\begin{frame}{boeing method}
\protect\hypertarget{boeing-method-3}{}
\begin{itemize}
\tightlist
\item
  The Boeing method is derived by separating the geometry effects from
  load and material effects in the Boeing-Walker equation.
\end{itemize}

\textbackslash{[}\frac{da}{dN} = \left[\frac{1}{n}\right]\frac{dL}{dN} =
10\^{}\{-4\} \left[\frac{k_{max}Z}{m_T}\right]\^{}p\textbackslash{]}

\textbackslash{[}\frac{dL}{dN} = n 10\^{}\{-4\}
\left[\frac{k_{max}Z}{m_T}\right]\^{}p\textbackslash{]}

\textbackslash{[}\frac{dN}{dL} = \frac{1}{n} 10\^{}\{4\}
\left[\frac{m_T}{k_{max}Z}\right]\^{}p\textbackslash{]}
\end{frame}

\begin{frame}{boeing method}
\protect\hypertarget{boeing-method-4}{}
\textbackslash{[}\int\emph{\{0\}\^{}\{N\}dN = \frac{10^{4}}{n}
\int}\{L\_0\}\^{}\{L\_f\} \left[\frac{m_T}{k_{max}Z}\right]\^{}p
dL\textbackslash{]}

\textbackslash{[}N = 10\^{}\{4\}
\left(\frac{m_t}{z\sigma_{max}}\right)\^{}p \int\_\{L\_0\}\^{}\{L\_f\}
\frac{dL}{\left( n\sqrt{\pi L/n}\beta\right)^p}\textbackslash{]}
\end{frame}

\begin{frame}{boeing method}
\protect\hypertarget{boeing-method-5}{}
\begin{itemize}
\tightlist
\item
  In this form, the term \textbackslash(10\^{}\{4\}
  \left(\frac{m_t}{z\sigma_{max}}\right)\^{}p\textbackslash) is strictly
  from the applied load and material, while
  \textbackslash(\int\_\{L\_0\}\^{}\{L\_f\}
  \frac{dL}{\left( n\sqrt{\pi L/n}\beta\right)^p}\textbackslash) is from
  geometry
\item
  If we now define \emph{G} to account for crack geometry
\end{itemize}

\textbackslash{[}G =
\left[ \int_{L_0}^{L_f} \frac{dL}{\left( n\sqrt{\pi L/n}\beta\right)^p} \right] \^{}\{-1/p\}\textbackslash{]}
\end{frame}

\begin{frame}{boeing method}
\protect\hypertarget{boeing-method-6}{}
\begin{itemize}
\tightlist
\item
  And define \textbackslash(z\sigma\_\{max\} = S\textbackslash) as the
  equivalent load spectrum, then we have
\end{itemize}

\textbackslash{[}N = 10\^{}4
\left(\frac{m_t/G}{S}\right)\^{}p\textbackslash{]}

\begin{itemize}
\tightlist
\item
  Using this method, \emph{G} is typically looked up from a chart (such
  as on p.~369)
\end{itemize}
\end{frame}

\begin{frame}{boeing method}
\protect\hypertarget{boeing-method-7}{}
\begin{itemize}
\tightlist
\item
  To replace a repeated load spectrum with an equivalent load, we need
  to invert the relationship
\item
  The previous equation~gives cycles per crack growth, inverting gives
  crack growth per cycle
\end{itemize}

\textbackslash{[}\text{crack growth per cycle} = 10\^{}\{-4\}
\left(\frac{m_t/G}{S}\right)\^{}\{-p\}\textbackslash{]}
\end{frame}

\begin{frame}{boeing method}
\protect\hypertarget{boeing-method-8}{}
\begin{itemize}
\tightlist
\item
  If we consider a general, repeatable ``block'', we have
\end{itemize}

\textbackslash{[}10\^{}\{-4\} \left( m\_t/G \right)\^{}\{-p\} \sum\_i
\left( \frac{1}{z\sigma_{max}} \right)\_i\^{}\{-p\} N\_i = 10\^{}\{-4\}
\left( \frac{m_t/G}{S} \right)\^{}\{-p\}\textbackslash{]}

\begin{itemize}
\tightlist
\item
  Which simplifies to
\end{itemize}

\textbackslash{[} \sum\emph{i (z \sigma}\{max\}) = S\^{}p
\textbackslash{]}
\end{frame}

\begin{frame}{boeing method example}
\protect\hypertarget{boeing-method-example}{}
\begin{itemize}
\tightlist
\item
  (from p.~366), \emph{q} = 0.6, \emph{p} = 3.9
\end{itemize}
\end{frame}

\begin{frame}{boeing method example - cont.}
\protect\hypertarget{boeing-method-example---cont.}{}
Count cycles from the right (instead of the left)
\end{frame}

\end{document}
