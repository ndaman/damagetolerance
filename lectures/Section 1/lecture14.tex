% !TeX program = xelatex
\documentclass[10pt]{beamer}

\usetheme{metropolis}

\usepackage{pgfplots}
\usepgfplotslibrary{fillbetween}
\usepackage{pgfopts}
\usepackage{amsmath}
\usepackage{structuralanalysis}
\usepackage{tikz}
\usepackage{tikz-3dplot}
\usepackage{chngcntr}
\usepackage{wasysym}
\usepackage{mathtools}
\usepackage{alphalph}
\usepackage{xcolor}
\usepackage[showdow=false, en-US]{datetime2}
\usepackage{hyperref}

\newcommand{\highlight}[1]{%
	\colorbox{red!50}{$\displaystyle#1$}}

\setcounter{lecture}{14}
\counterwithin{equation}{lecture}
\makeatletter
\def\user@resume{resume}
\def\user@intermezzo{intermezzo}
%
\newcounter{previousequation}
\newcounter{lastsubequation}
\newcounter{savedparentequation}
\setcounter{savedparentequation}{1}
% 
\renewenvironment{subequations}[1][]{%
	\def\user@decides{#1}%
	\setcounter{previousequation}{\value{equation}}%
	\ifx\user@decides\user@resume 
	\setcounter{equation}{\value{savedparentequation}}%
	\else  
	\ifx\user@decides\user@intermezzo
	\refstepcounter{equation}%
	\else
	\setcounter{lastsubequation}{0}%
	\refstepcounter{equation}%
	\fi\fi
	\protected@edef\theHparentequation{%
		\@ifundefined {theHequation}\theequation \theHequation}%
	\protected@edef\theparentequation{\theequation}%
	\setcounter{parentequation}{\value{equation}}%
	\ifx\user@decides\user@resume 
	\setcounter{equation}{\value{lastsubequation}}%
	\else
	\setcounter{equation}{0}%
	\fi
	\def\theequation  {\theparentequation  \alph{equation}}%
	\def\theHequation {\theHparentequation \alph{equation}}%
	\ignorespaces
}{%
%  \arabic{equation};\arabic{savedparentequation};\arabic{lastsubequation}
\ifx\user@decides\user@resume
\setcounter{lastsubequation}{\value{equation}}%
\setcounter{equation}{\value{previousequation}}%
\else
\ifx\user@decides\user@intermezzo
\setcounter{equation}{\value{parentequation}}%
\else
\setcounter{lastsubequation}{\value{equation}}%
\setcounter{savedparentequation}{\value{parentequation}}%
\setcounter{equation}{\value{parentequation}}%
\fi\fi
%  \arabic{equation};\arabic{savedparentequation};\arabic{lastsubequation}
\ignorespacesafterend
}
\makeatother
\title{AE 737 - Mechanics of Damage Tolerance}
\subtitle{Lecture \arabic{lecture}}
\date{Last Updated: \today\ at \DTMcurrenttime}
\author{Dr. Nicholas Smith}
\institute{Wichita State University, Department of Aerospace Engineering}
% \titlegraphic{\hfill\includegraphics[height=1.5cm]{logo/logo}}

\begin{document}

\maketitle
\begin{frame}{schedule}
	\begin{itemize}
		\item 10 Mar - Exam return, Final Project discussion, Project abstract assigned
		\item 22 Mar - Stress based fatigue, Homework 6 assigned
		\item 24 Mar - Stress based fatigue
		\item 29 Mar - Influence of notches on fatigue, Homework 7 assigned, Homework 6 due
		\item 31 Mar - Strain based fatigue, project abstract due
	\end{itemize}
\end{frame}

\begin{frame}
  \frametitle{outline}
  \setbeamertemplate{section in toc}[sections numbered]
  \tableofcontents[hideallsubsections]
\end{frame}

\section{exam}

%statistics on exam grades, curve
\begin{frame}{exam}
	\begin{itemize}[<+->]
		\item Before curve, average score: 80.3
		\item Before curve, std dev: 15.8
		\item High score: 96 (1 student, 3 had 95)
	\end{itemize}
\end{frame}

\begin{frame}{exam}
	\begin{itemize}[<+->]
		\item Purpose of curve: tighten distribution (goal is to have std dev = 10\%)
		\item I find one exam that I consider excellent and set that as the high score (95 in this case)
		\item I find one that I consider "average" and set that to be a C (61 in this case)
		\item Curve for this exam: Adjusted score = 0.735 (original score) + 30.15
		\item New average: 89.2, new std dev: 11.6
	\end{itemize}
\end{frame}

\begin{frame}{before curve}
	
	\begin{tikzpicture}
	\begin{axis}[
	ybar,
	ymin=0,
	ymax=12
	]
	\addplot +[
	hist={
		bins=7,
		data min=50,
		data max=100
	}   
	] table [y index=0] {unadjusted.csv};
	\end{axis}
	\end{tikzpicture}
\end{frame}

\begin{frame}{after curve}
	
	\begin{tikzpicture}
	\begin{axis}[
	ybar,
	ymin=0,
	ymax=12,
	]
	\addplot +[
	hist={
		bins=7,
		data min=50,
		data max=100
	}   
	] table [y index=0] {adjusted.csv};
	\end{axis}
	\end{tikzpicture}
\end{frame}

\begin{frame}<handout:0>{solutions}
	on board
\end{frame}

\section{final project}

\begin{frame}{general description}
	\begin{itemize}[<+->]
		\item This is in place of a final exam
		\item Should demonstrate your understanding of the course as a whole
		\item Choose any real object
		\item Needs to undergo some cyclic loading (for fatigue)
		\item Materials, loads, and any other "given" data can be made up
	\end{itemize}
\end{frame}

\begin{frame}{overview}
	\begin{itemize}[<+->]
		\item Estimate stress intensity factor at some critical location
		\item Estimate residual strength (use a "typical" crack length)
		\item Estimate crack growth and propagation (fatigue)
		\item Suggest reasonable inspection cycle for safe use
		\item Suggest an improvement to make part more damage tolerant
	\end{itemize}
\end{frame}

\begin{frame}{grade breakdown}
	\begin{itemize}[<+->]
		\item Per course syllabus, project will be worth 25\% of final grade
		\item 5\% Project abstract submission and approval
		\item 10\% for each major component
		\begin{itemize}
			\item stress intensity factor
			\item residual strength
			\item fatigue (x2, growth and propagation)
			\item inspection cycle
		\end{itemize}
		\item 20\% for damage tolerant improvement
		\item 20\% general presentation, organization, and grammar
	\end{itemize}
\end{frame}

\begin{frame}{project abstract}
	\begin{itemize}[<+->]
		\item Main purpose of abstract is for you to make sure your idea fits with project purpose
		\item I will give you feedback on how to tweak your proposed idea to better meet project purpose
		\item Abstract submission should be 1-2 pages
		\item Briefly describe your chosen part, how it undergoes cyclic loading, what location you intend to consider for the stress intensity factor.
		\item This is like a proposal: convince me that your idea has what it takes to be a great final project
	\end{itemize}
\end{frame}

\begin{frame}{justify assumptions}
	\begin{itemize}[<+->]
		\item You will need to make many assumptions in order to complete this project
		\item Clearly state your assumptions and justify them (i.e. if you assume plane strain conditions, justify that by showing how thick your part is)
		\item Although will not have experimental or FE analysis specific to your part, use concepts from other data in the text (stiffeners, multiple site damage) in a qualitative manner
	\end{itemize}
\end{frame}

\begin{frame}{figures}
	\begin{itemize}[<+->]
		\item Figures can greatly enhance your project report, if you use them well
		\item Many readers will jump to figures in a report, include sufficient information in caption and axis labels so a reader with general damage tolerance understanding can understand your figure
		\item This will interest them in the rest of your paper
	\end{itemize}
\end{frame}

\end{document}
