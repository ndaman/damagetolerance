% !TeX program = xelatex
\documentclass[10pt]{beamer}

\usetheme{metropolis}

\usepackage{pgfplots}
\usepgfplotslibrary{fillbetween}
\usepackage{pgfopts}
\usepackage{amsmath}
\usepackage{structuralanalysis}
\usepackage{tikz}
\usepackage{tikz-3dplot}
\usepackage{chngcntr}
\usepackage{wasysym}
\usepackage{mathtools}
\usepackage{alphalph}
\usepackage{xcolor}
\usepackage[showdow=false, en-US]{datetime2}
\usepackage{hyperref}

\newcommand{\highlight}[1]{%
	\colorbox{red!50}{$\displaystyle#1$}}

\setcounter{lecture}{21}
\counterwithin{equation}{lecture}
\makeatletter
\def\user@resume{resume}
\def\user@intermezzo{intermezzo}
%
\newcounter{previousequation}
\newcounter{lastsubequation}
\newcounter{savedparentequation}
\setcounter{savedparentequation}{1}
% 
\renewenvironment{subequations}[1][]{%
	\def\user@decides{#1}%
	\setcounter{previousequation}{\value{equation}}%
	\ifx\user@decides\user@resume 
	\setcounter{equation}{\value{savedparentequation}}%
	\else  
	\ifx\user@decides\user@intermezzo
	\refstepcounter{equation}%
	\else
	\setcounter{lastsubequation}{0}%
	\refstepcounter{equation}%
	\fi\fi
	\protected@edef\theHparentequation{%
		\@ifundefined {theHequation}\theequation \theHequation}%
	\protected@edef\theparentequation{\theequation}%
	\setcounter{parentequation}{\value{equation}}%
	\ifx\user@decides\user@resume 
	\setcounter{equation}{\value{lastsubequation}}%
	\else
	\setcounter{equation}{0}%
	\fi
	\def\theequation  {\theparentequation  \alph{equation}}%
	\def\theHequation {\theHparentequation \alph{equation}}%
	\ignorespaces
}{%
%  \arabic{equation};\arabic{savedparentequation};\arabic{lastsubequation}
\ifx\user@decides\user@resume
\setcounter{lastsubequation}{\value{equation}}%
\setcounter{equation}{\value{previousequation}}%
\else
\ifx\user@decides\user@intermezzo
\setcounter{equation}{\value{parentequation}}%
\else
\setcounter{lastsubequation}{\value{equation}}%
\setcounter{savedparentequation}{\value{parentequation}}%
\setcounter{equation}{\value{parentequation}}%
\fi\fi
%  \arabic{equation};\arabic{savedparentequation};\arabic{lastsubequation}
\ignorespacesafterend
}
\makeatother
\title{AE 737 - Mechanics of Damage Tolerance}
\subtitle{Lecture \arabic{lecture}}
\date{Last Updated: \today\ at \DTMcurrenttime}
\author{Dr. Nicholas Smith}
\institute{Wichita State University, Department of Aerospace Engineering}
% \titlegraphic{\hfill\includegraphics[height=1.5cm]{logo/logo}}

\begin{document}

\maketitle
\begin{frame}{schedule}
	\begin{itemize}
		\item 12 Apr - Retardation, Boeing Commercial Method
		\item 14 Apr - Exam Review, Homework 8 Due
		\item 19 Apr - Damage Tolerance
		\item 21 Apr - Exam 2
		\item 26 Apr - Exam Solutions, Damage Tolerance
		\item 28 Apr - SPTE, AFGROW, Finite Elements
%		\item 3 May - Finite Elements
%		\item 5 May - Non-Destructive Testing, Composites, Final Project Due May 10
	\end{itemize}
\end{frame}

\begin{frame}
  \frametitle{outline}
  \setbeamertemplate{section in toc}[sections numbered]
  \tableofcontents[hideallsubsections]
\end{frame}

\section{review}

\begin{frame}{numerical algorith}
	\begin{itemize}[<+->]
		\item While the Paris Law can be integrated directly (for simple load cases), many of the other formulas cannot
		\item A simple numerical algorithm for determining incremental crack growth is
		\item[]	\begin{equation}
		a_{i+1} = a_i + \left(\frac{da}{dN}\right)_i\left(\Delta N\right)_i
		\end{equation}
		\item This method is quite tedious by hand (need many $a_i$ values for this to be accurate) 
		\item But is simple to do in Excel, MATLAB, Python, or many other codes
		\item For most accurate results, use $\Delta N = 1$, but this is often unnecessary
		\item When trying to use large $\Delta N$, check convergence by using larger and smaller $\Delta N$ values
	\end{itemize}
\end{frame}

%TODO - Convergence example
\begin{frame}{convergence example}
	
\end{frame}

\begin{frame}{variable load cases}
	\begin{itemize}[<+->]
		\item In practice variable loads are often seen
		\item The most basic way to handle these is to simply calculate the crack length after each block of loading
		\item We will discuss an alternate method, which is more convenient for flight "blocks" next class
		\item We will also discuss "retardation" models next class
	\end{itemize}
\end{frame}

\section{boeing method}

\begin{frame}{boeing method for variable amplitude loads}
	\begin{itemize}[<+->]
		\item Whether integrating numerically or analytically, it is time-consuming to consider multiple repeated loads
		\item It is particularly difficult to consider flight loads, which can vary by "mission"
		\item For example, an aircraft may fly three different routes, in no particular order, but with a known percentage of time spent in each route
		\item Traditional methods would use a random mix of each load spectra
		\item The Boeing Method combines each repeatable load spectrum into one single equivalent cycle
	\end{itemize}
\end{frame}

\begin{frame}{boeing method}
	\begin{itemize}[<+->]
		\item The Boeing method is derived by separating the geometry effects from load and material effects in the Boeing-Walker equation.
		\item[] \begin{equation}
		\frac{da}{dN} = \left[\frac{1}{n}\right]\frac{dL}{dN} = 10^{-4} \left[\frac{k_{max}Z}{m_T}\right]^p
		\end{equation}
		\item[] \begin{equation}
		\frac{dL}{dN} = n 10^{-4} \left[\frac{k_{max}Z}{m_T}\right]^p
		\end{equation}
		\item[] \begin{equation}
		\frac{dN}{dL} = \frac{1}{n} 10^{4} \left[\frac{m_T}{k_{max}Z}\right]^p
		\end{equation}
		\item[] \begin{equation}
		\int_{0}^{N}dN = \frac{10^{4}}{n}  \int_{L_0}^{L_f} \left[\frac{m_T}{k_{max}Z}\right]^p dL
		\end{equation}
		\item[] \begin{equation}
		N = 10^{4} \left(\frac{m_t}{z\sigma_{max}}\right)^p  \int_{L_0}^{L_f} \frac{dL}{\left( n\sqrt{\pi L/n}\beta\right)^p}
		\end{equation}
	\end{itemize}
\end{frame}

\begin{frame}{boeing method}
	\begin{itemize}[<+->]
		\item In this form, the term $10^{4} \left(\frac{m_t}{z\sigma_{max}}\right)^p$ is strictly from the applied load and material, while $\int_{L_0}^{L_f} \frac{dL}{\left( n\sqrt{\pi L/n}\beta\right)^p}$ is from geometry
		\item If we now define $G$ to account for crack geometry
		\item[] \begin{equation}
		G = \left[ \int_{L_0}^{L_f} \frac{dL}{\left( n\sqrt{\pi L/n}\beta\right)^p} \right] ^{-1/p}
		\end{equation}
		\item And define $z \sigma_{max} = S$ as the equivalent load spectrum, then we have
		\item[] \begin{equation}
		\label{eq:boeing}
		N = 10^4 \left(\frac{m_t/G}{S}\right)^p
		\end{equation}
		\item Using this method, $G$ is typically looked up from a chart (such as on p. 369)
	\end{itemize}
\end{frame}

\begin{frame}{boeing method}
	\begin{itemize}[<+->]
		\item To replace a repeated load spectrum with an equivalent load, we need to invert the relationship
		\item Equation~\ref{eq:boeing} gives cycles per crack growth, inverting gives crack growth per cycle
		\item[] \begin{equation}
		\text{crack growth per cycle} = 10^{-4} \left(\frac{m_t/G}{S}\right)^{-p}
		\end{equation}
		\item If we consider a general, repeatable "block", we have
		\item[] \begin{equation}
		10^{-4} \left( m_t/G \right)^{-p} \sum_i \left( \frac{1}{z\sigma_{max}} \right)_i^{-p} N_i = 10^{-4} \left( \frac{m_t/G}{S} \right)^{-p}
		\end{equation}
		\item Which simplifies to
		\begin{equation}
		\sum_i \left( z\sigma_{max} \right)_i^{p}N_i = \left( S \right)^{p}
		\end{equation}		
	\end{itemize}
\end{frame}

\begin{frame}{boeing method example}
\end{frame}

\begin{frame}{boeing method example - cont.}
	cycle counting
\end{frame}

\section{cycle counting}

\begin{frame}{cycle counting}
	\begin{itemize}[<+->]
		\item As illustrated in our previous example, cycle counting method can make a difference for variable amplitude loads
		\item Two common methods for cycle counting that give similar results are known as the "rainflow" and "range-pair" methods
		\item ASTM E1049-85 "Standard Practices for Cycle Counting in Fatigue Analysis"
	\end{itemize}
\end{frame}

\begin{frame}{rain-flow method}
	\begin{enumerate}[<+->]
		\item Read next peak or valley. $S$ is the starting point, $Y$ is the first range, $X$ is the second range
		\item If $X < Y$ advance points ($S$ remains same, $Y$ and $X$ change)
		\item If $X \ge Y$ and $Y$ contains $S$, count $Y$ as 1/2-cycle, discard $S$ and go to 1
		\item If $X \ge Y$ and $Y$ does not contain $S$, count $Y$ as 1 cycle, discard both points in $Y$ and go to 1 ($S$ remains same)
		\item When end of data is reached, count each range as 1/2-cycle
	\end{enumerate}
\end{frame}

\begin{frame}{range-pair method}
	\begin{enumerate}[<+->]
		\item Read next peak or valley. $Y$ is the first range, $X$ is the second range
		\item If $X < Y$ advance points
		\item If $X \ge Y$, count $Y$ as 1 cycle and discard both points in $Y$, go to 1
		\item Remaining cycles are counted backwards from end of history
	\end{enumerate}
\end{frame}

\begin{frame}{cycle counting example}
	
\end{frame}

\section{crack growth retardation}

\begin{frame}{crack growth retardation}
	\begin{itemize}[<+->]
		\item When an overload is applied, the plastic zone is larger
		\item This zone has residual compressive stresses, which slow crack growth until the crack grows beyond this over-sized plastic zone
		\item We will discuss three retardation models, but no model has been shown to be perfect in all cases
		\item The Wheeler method reduces $da/dN$, the Willenborg model reduces $\Delta K$, and the Closure model increases $R$ (increases $\sigma_{min}$)
	\end{itemize}
\end{frame}

\begin{frame}{wheeler retardation}
	\begin{itemize}[<+->]
		\item As long as crack is within overload plastic zone, we scale $da/dN$ by some $\phi$
		\item[] \begin{equation}
		(a_i + r_{pi}) = (a_{ol} + r_{pol})
		\end{equation}
		\item And $\phi$ is given by
		\begin{equation}
		\phi_i = \left[\frac{r_{pi}}{a_{ol}+r_{pol}-a_i}\right]^m
		\end{equation}
		\item and the constant, $m$ is to be determined experimentally
	\end{itemize}
\end{frame}

\begin{frame}{wheeler example}
	
\end{frame}

\begin{frame}{willenborg retardation}
	\begin{itemize}[<+->]
		\item Once again, we consider that retardation occurs when $(a_i + r_{pi}) = (a_{ol} + r_{pol})$
		\item Willenborg assumes that the residual compressive stress in the plastic zone creates an effective, $K_{max,eff}$, where $K_{max,eff} = K_{max} - K_{comp}$
		\item The effective stress intensity factor is given by
		\begin{equation}
		K_{max,eff} = K_{max,i} - \left[K_{max,OL}\sqrt{1-\frac{\Delta a_i}{r_{pol}}} - K_{max,i} \right]
		\end{equation} 
	\end{itemize}
\end{frame}

\begin{frame}{gallagher and hughes correction}
	\begin{itemize}[<+->]
		\item Galagher and Hughes observed that the Willenborg model stops cracks when they still propagate
		\item They proposed a correction to the model
		\item[] \begin{equation}
		K_{max,eff} = K_{max,i} - \phi_i\left[K_{max,OL}\sqrt{1-\frac{\Delta a_i}{r_{pol}}} - K_{max,i} \right]
		\end{equation} 
		\item And the correction factor, $\phi_i$ is given by
		\begin{equation}
		\phi_i \frac{1-K_{TH}/K_{max,i}}{s_{ol}-1}
		\end{equation}
	\end{itemize}
\end{frame}

\begin{frame}{willenborg example}
	
\end{frame}

\begin{frame}{closure model}
	\begin{itemize}[<+->]
		\item Once again, we consider that retardation occurs when $(a_i + r_{pi}) = (a_{ol} + r_{pol})$
		\item Within the overloaded plastic zone, the opening stress required can be expressed as
		\begin{equation}
		\sigma_{OP} = \sigma_{max} (1-(1-C_{f0})(1+0.6R)(1-R))
		\end{equation}
		\item Commonly this is expressed using the Closure Factor, $C_f$
		\item[] \begin{equation}
		C_f = \frac{\sigma_{OP}}{\sigma_{max}} = (1-(1-C_{f0})(1+0.6R)(1-R))
		\end{equation}
		\item Where $C_{f0}$ is the value of the Closure Factor at $R=0$
	\end{itemize}
\end{frame}

\begin{frame}{closure model}
	\begin{itemize}[<+->]
		\item When using the closure model, we replace $R$ with $C_f$
		\item If the model we are using is in terms of $\Delta K$ we will also need to use $\Delta K = (1-C_f) K_{max}$
	\end{itemize}
\end{frame}

\begin{frame}{closure example}
	
\end{frame}

\begin{frame}{compressive under-loads}
	\begin{itemize}[<+->]
		\item Just as a tensile "overload" retards crack growth, we might expect a compressive "underload" to accelerate crack growth
		\item This effect is not usually modeled for a few reasons
		\begin{enumerate}
			\item Compressive underloads are uncommon in airframes
			\item The acceleration effect is minimal
			\item Analysis is generally adjusted with experimental data, so acceleration can be built in to current model
			\item Structures with large compressive loads are not generally subject to crack propagation problems
		\end{enumerate}
	\end{itemize}
\end{frame}
\end{document}
