\documentclass[11pt, a4paper]{article}
%\usepackage{geometry}
\usepackage[inner=1.5cm,outer=1.5cm,top=2.5cm,bottom=2.5cm]{geometry}
\pagestyle{empty}
\usepackage{graphicx}
\usepackage{fancyhdr, lastpage, bbding, pmboxdraw}
\usepackage[usenames,dvipsnames]{color}
\usepackage{float}
\definecolor{darkblue}{rgb}{0,0,.6}
\definecolor{darkred}{rgb}{.7,0,0}
\definecolor{darkgreen}{rgb}{0,.6,0}
\definecolor{red}{rgb}{.98,0,0}
\usepackage[colorlinks,pagebackref,pdfusetitle,urlcolor=darkblue,citecolor=darkblue,linkcolor=darkred,bookmarksnumbered,plainpages=false]{hyperref}
\renewcommand{\thefootnote}{\fnsymbol{footnote}}

\pagestyle{fancyplain}
\fancyhf{}
\lhead{ \fancyplain{}{Course Name} }
%\chead{ \fancyplain{}{} }
\rhead{ \fancyplain{}{\today} }
%\rfoot{\fancyplain{}{page \thepage\ of \pageref{LastPage}}}
\fancyfoot[RO, LE] {page \thepage\ of \pageref{LastPage} }
\thispagestyle{plain}

%%%%%%%%%%%% LISTING %%%
\usepackage{listings}
\usepackage{caption}
\DeclareCaptionFont{white}{\color{white}}
\DeclareCaptionFormat{listing}{\colorbox{gray}{\parbox{\textwidth}{#1#2#3}}}
\captionsetup[lstlisting]{format=listing,labelfont=white,textfont=white}
\usepackage{verbatim} % used to display code
\usepackage{fancyvrb}
\usepackage{acronym}
\usepackage{amsthm}
\VerbatimFootnotes % Required, otherwise verbatim does not work in footnotes!



\definecolor{OliveGreen}{cmyk}{0.64,0,0.95,0.40}
\definecolor{CadetBlue}{cmyk}{0.62,0.57,0.23,0}
\definecolor{lightlightgray}{gray}{0.93}



\lstset{
%language=bash,                          % Code langugage
basicstyle=\ttfamily,                   % Code font, Examples: \footnotesize, \ttfamily
keywordstyle=\color{OliveGreen},        % Keywords font ('*' = uppercase)
commentstyle=\color{gray},              % Comments font
numbers=left,                           % Line nums position
numberstyle=\tiny,                      % Line-numbers fonts
stepnumber=1,                           % Step between two line-numbers
numbersep=5pt,                          % How far are line-numbers from code
backgroundcolor=\color{lightlightgray}, % Choose background color
frame=none,                             % A frame around the code
tabsize=2,                              % Default tab size
captionpos=t,                           % Caption-position = bottom
breaklines=true,                        % Automatic line breaking?
breakatwhitespace=false,                % Automatic breaks only at whitespace?
showspaces=false,                       % Dont make spaces visible
showtabs=false,                         % Dont make tabls visible
columns=flexible,                       % Column format
morekeywords={__global__, __device__},  % CUDA specific keywords
}

%%%%%%%%%%%%%%%%%%%%%%%%%%%%%%%%%%%%
\begin{document}
\begin{center}
{\Large \textsc{AE 737 - Mechanics of Damage Tolerance}}
\end{center}
\begin{center}
Spring 2016
\end{center}
%\date{September 26, 2014}

\begin{center}
\rule{6in}{0.4pt}
\begin{minipage}[t]{.75\textwidth}
\begin{tabular}{llcccll}
\textbf{Instructor:} & Dr. Nicholas A Smith & & &  & \textbf{Time:} & TR 4:10 -- 5:25 pm \\ %75 min
\textbf{Email:} &  \href{mailto:Nicholas.A.Smith@wichita.edu}{Nicholas.A.Smith@wichita.edu} & & & & \textbf{Place:} & 209 Wallace Hall\end{tabular}
\end{minipage}
\rule{6in}{0.4pt}
\end{center}
\vspace{.5cm}
\setlength{\unitlength}{1in}
\renewcommand{\arraystretch}{2}

%
%\noindent\textbf{Course Pages:} \begin{enumerate}
%\item \url{http://yourWebPage1.com/teaching}
%\item \url{http://yourWebPage2.com/teaching}
%\end{enumerate}

\vskip.15in
\noindent\textbf{Office Hours:} TBD, 204 Wallace Hall

\vskip.15in
\noindent\textbf{Course Textbook:} %\footnotemark
The text used for this class is made up of notes originally assembled by Dr. Bert L. Smith, which will be prepared by the department. 
You can pick up your copy of the text in the AE offices (WH 200), the printing cost for this material is \$25 (cash or check only).

\vskip.15in
\noindent\textbf{Other References:}
The notes we use in this course provide a very good base, but sometimes supplemental material is beneficial.
The following texts are recommended as additional references:
\begin{itemize}
\item H. Broek, {\textit{The Practical Use of Fracture Mechanics}}
\item H.L. Ewalds and R.J.H. Wanhill, {\textit{Fracture Mechanics}}
\item A. F. Grandt, {\textit{Fundamentals of Structural Integrity}}
\end{itemize} 

% \footnotetext{Downloadable ebook versions are available on AeLP.}

\vskip.15in
\noindent\textbf{Objectives:}  In this course we will use concepts from fracture mechanics. 
Students will be asked to determine residual strength and predict crack growth in structures under fatigue loading.

\vskip.15in
\noindent\textbf{Prerequisites:}
Instructor's consent (previous experience in solid mechanics, AE333 and AE525, and differential equations, AE555, is advised).

\vspace*{.15in}

\noindent \textbf{Tentative Course Outline:}
\begin{center} 
\begin{minipage}{5in}
\begin{flushleft}
Stress Intensity \\
Plastic Zone \\
Fracture Toughness \\
Residual Strength \\
Exam 1 \\
Fatigue Crack Growth Rate \\
Fatigue Crack Propagation \\
Exam 2 \\
Damage Tolerance
\end{flushleft}
\end{minipage}
\end{center}


\vspace*{.15in}
\noindent\textbf{Homework:} Homework may be submitted either electronically before the class period it is due or in class on the due date.
It is anticipated that one homework assignment per lecture block (as given in the course outline) will be assigned.
To be graded, homework must be legible and organized in a manner that makes it easy for me to follow your work.
Late homework will not be accepted, homework problems submitted out of order will not be graded.

\vspace*{.15in}
\noindent\textbf{Exams:} There will be two major midterm exams during the semester.
Exams will be closed-book and closed-notes, but there will be an equation sheet provided.

\vspace*{.15in}
\noindent\textbf{Final Project:} More information on the final project will be provided at a later time in the course.
In this final project you will be required to perform residual strength, fatigue, and damage tolerance analysis on a real-life part of your choosing.
You will use the principles developed in this class to estimate the maximum load your part can carry, a reasonable inspection cycle, etc.
The part you choose should undergo cyclic loading of some form for a fatigue analysis.

\vspace*{.15in}
\noindent\textbf{Grading Policy:} Homework (15\%),  Midterm 1 (30\%), Midterm 2 (30\%), Final Project (25\%). 
Final grades follow a traditional scale of:
%%insert grading table
\begin{table}[H]
\begin{tabular}{cccccccccccc}
 A & A- & B+ & B & B- & C+ & C & C- & D+ & D & D- & F \\ 
 93-100 & 90-93 & 87-90 & 83-87 & 80-83 & 77-80 & 73-77 & 70-73 & 67-70 & 63-67 & 60-63 & 0-60 \\ 
\end{tabular}
\end{table}
Per department policy, final course grades will not be disclosed before the official notifications by the University.

\vskip.15in
\noindent\textbf{Academic Honesty:}   Lack of knowledge of the academic honesty policy is not a reasonable explanation for a violation.
You are expected to submit your own work, but this does not exclude working and studying in groups.
Group study is encouraged, but be sure that you submit your own work.

%%%%%% THE END 
\end{document} 