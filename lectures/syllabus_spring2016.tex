\documentclass[11pt, a4paper]{article}
%\usepackage{geometry}
\usepackage[inner=1.5cm,outer=1.5cm,top=2.5cm,bottom=2.5cm]{geometry}
\pagestyle{empty}
\usepackage{graphicx}
\usepackage{fancyhdr, lastpage, bbding, pmboxdraw}
\usepackage[usenames,dvipsnames]{color}
\usepackage{float}
\usepackage{caption}
\definecolor{darkblue}{rgb}{0,0,.6}
\definecolor{darkred}{rgb}{.7,0,0}
\definecolor{darkgreen}{rgb}{0,.6,0}
\definecolor{red}{rgb}{.98,0,0}
\usepackage[colorlinks,pagebackref,pdfusetitle,urlcolor=darkblue,citecolor=darkblue,linkcolor=darkred,bookmarksnumbered,plainpages=false]{hyperref}
\renewcommand{\thefootnote}{\fnsymbol{footnote}}

\pagestyle{fancyplain}
\fancyhf{}
\lhead{ \fancyplain{}{Mechanics of Damage Tolerance} }
%\chead{ \fancyplain{}{} }
\rhead{ \fancyplain{}{\today} }
%\rfoot{\fancyplain{}{page \thepage\ of \pageref{LastPage}}}
\fancyfoot[RO, LE] {page \thepage\ of \pageref{LastPage} }
\thispagestyle{plain}

%%%%%%%%%%%% LISTING %%%
\usepackage{listings}
\usepackage{caption}
\DeclareCaptionFont{white}{\color{white}}
\DeclareCaptionFormat{listing}{\colorbox{gray}{\parbox{\textwidth}{#1#2#3}}}
\captionsetup[lstlisting]{format=listing,labelfont=white,textfont=white}
\usepackage{verbatim} % used to display code
\usepackage{fancyvrb}
\usepackage{acronym}
\usepackage{amsthm}
\VerbatimFootnotes % Required, otherwise verbatim does not work in footnotes!



\definecolor{OliveGreen}{cmyk}{0.64,0,0.95,0.40}
\definecolor{CadetBlue}{cmyk}{0.62,0.57,0.23,0}
\definecolor{lightlightgray}{gray}{0.93}



\lstset{
%language=bash,                          % Code langugage
basicstyle=\ttfamily,                   % Code font, Examples: \footnotesize, \ttfamily
keywordstyle=\color{OliveGreen},        % Keywords font ('*' = uppercase)
commentstyle=\color{gray},              % Comments font
numbers=left,                           % Line nums position
numberstyle=\tiny,                      % Line-numbers fonts
stepnumber=1,                           % Step between two line-numbers
numbersep=5pt,                          % How far are line-numbers from code
backgroundcolor=\color{lightlightgray}, % Choose background color
frame=none,                             % A frame around the code
tabsize=2,                              % Default tab size
captionpos=t,                           % Caption-position = bottom
breaklines=true,                        % Automatic line breaking?
breakatwhitespace=false,                % Automatic breaks only at whitespace?
showspaces=false,                       % Dont make spaces visible
showtabs=false,                         % Dont make tabls visible
columns=flexible,                       % Column format
morekeywords={__global__, __device__},  % CUDA specific keywords
}

%%%%%%%%%%%%%%%%%%%%%%%%%%%%%%%%%%%%
\begin{document}
\begin{center}
{\Large \textsc{AE 737 - Mechanics of Damage Tolerance}}
\end{center}
\begin{center}
Spring 2016
\end{center}
%\date{September 26, 2014}

\begin{center}
\rule{6.5in}{0.4pt}
\begin{minipage}[t]{.75\textwidth}
\hspace*{-1.2cm}\begin{tabular}{llcccll}
\textbf{Instructor:} & Dr. Nicholas A Smith & & &  & \textbf{Time:} & TR 4:10 -- 5:25 pm \\ %75 min
\textbf{Department:} & Aerospace Engineering & & & & \textbf{Place:} & 209 Wallace Hall \\
\textbf{Email:} &  \href{mailto:Nicholas.A.Smith@wichita.edu}{Nicholas.A.Smith@wichita.edu} & & & & \textbf{Office:} & 204 Wallace Hall\\
\textbf{Phone:} & (316) 978-3307 & & & & \textbf{Office Hours:} & Fri 3:00 -- 5:00 pm\end{tabular}
\end{minipage}
\rule{6.5in}{0.4pt}
\end{center}
\vspace{.5cm}
\setlength{\unitlength}{1in}
\renewcommand{\arraystretch}{2}

%
%\noindent\textbf{Course Pages:} \begin{enumerate}
%\item \url{http://yourWebPage1.com/teaching}
%\item \url{http://yourWebPage2.com/teaching}
%\end{enumerate}

\vskip.15in
\noindent\textbf{How to use this syllabus:} 
This syllabus provides you with information specific to this course, and it also provides information about important university policies.  
This document should be viewed as a course overview; it is not a contract and is subject to change as the semester evolves.  
Any changes to the syllabus will be uploaded to Blackboard and e-mailed to all students (at their e-mail address listed on Blackboard, make sure this is up-to-date).

\vskip.15in
\noindent\textbf{Academic Honesty:} %\footnotemark
Students are responsible for knowing and following the Student Code of Conduct \url{http://webs.wichita.edu/inaudit/ch8_05.htm} and the Student Academic Honesty policy \url{http://webs.wichita.edu/inaudit/ch2_17.htm}.

\vskip.15in
\noindent\textbf{Course Description:}
The primary objective of this course is to introduce and apply the principles of fracture mechanics and fatigue analysis.
Topics covered include Stress Intensity, Residual Strength, Fatigue Analysis, Crack Propagation Analysis, and Damage Tolerance.

\vskip.15in
\noindent\textbf{Definition of a Credit Hour:}
Success in this 3 credit hour course is based on the expectation that students will spend, for each unit of credit, a minimum of 45 hours over the length of the course (normally 3 hours per unit per week with 1 of the hours used for lecture) for instruction and preparation/studying or course related activities for a total of 135 hours.

\vskip.15in
\noindent\textbf{Measurable Student Learning Outcomes:}
Upon successful completion of this course, students will be able to
\begin{itemize}
	\item Calculate stress intensity factors for a wide variety of geometry and loading conditions
	\item Analyze the residual strength of a damaged structure
	\item Calculate fatigue life and crack propagation
	\item Design structures for damage tolerance
\end{itemize}

\vskip.15in
\noindent\textbf{Course Textbook:} %\footnotemark
The text used for this class is made up of notes originally assembled by Dr. Bert L. Smith and Dr. Walter J. Horn, which will be prepared by the department. 
You can pick up your copy of the text in the AE offices (WH 200), the printing cost for this material is \$25 (cash or check only).

\vskip.15in
\noindent\textbf{Other References:}
The notes we use in this course provide a very good base, but sometimes supplemental material is beneficial.
The following texts are recommended as additional references:
\begin{itemize}
\item H. Broek, {\textit{The Practical Use of Fracture Mechanics}}
\item H.L. Ewalds and R.J.H. Wanhill, {\textit{Fracture Mechanics}}
\item A. F. Grandt, {\textit{Fundamentals of Structural Integrity}}
\end{itemize} 

\vskip.15in
\noindent\textbf{Prerequisites:}
Instructor's consent (previous experience in solid mechanics, AE333 and AE525, and differential equations, AE555, is advised).

\vspace*{.15in}
\noindent\textbf{Grading Policy:} 
Homework (15\%),  Midterm 1 (30\%), Midterm 2 (30\%), Final Project (25\%). 
Final grades follow a traditional scale of:
%%insert grading table
\begin{table}[H]
	\begin{tabular}{cccccccccccc}
		A & A- & B+ & B & B- & C+ & C & C- & D+ & D & D- & F \\ 
		93-100 & 90-93 & 87-90 & 83-87 & 80-83 & 77-80 & 73-77 & 70-73 & 67-70 & 63-67 & 60-63 & 0-60 \\ 
	\end{tabular}
\end{table}
Per department policy, final course grades will not be disclosed before the official notifications by the University.

\vspace*{.15in}
\noindent\textbf{Homework:} 
Homework may be submitted either electronically before the class period it is due or in class on the due date.
There will be a total of 8 Homework assignments, each worth 100 points.
Tentative homework due dates are given in the course schedule.
Late homework will not be accepted

\vspace*{.15in}
\noindent\textbf{Exams:} 
There will be two major midterm exams during the semester.
Exams will be closed-book and closed-notes, but there will be an equation sheet provided.
Anticipated exam dates are given the course schedule.

\vspace*{.15in}
\noindent\textbf{Final Project:} 
In lieu of a final exam we have a final project in this course.
In this final project you will be required to perform residual strength, fatigue, and damage tolerance analysis on a real-life part of your choosing.
You will use the principles developed in this class to estimate the maximum load your part can carry, a reasonable inspection cycle, etc.
The part you choose should undergo cyclic loading of some form for a fatigue analysis.
The final project will be due on May 10 by 5:00 pm.

\vspace*{.15in}
\begin{table}
	\captionsetup{singlelinecheck=false,justification=raggedright}
	\caption*{\noindent \textbf{Tentative Course Schedule:}}
	\centering
		\begin{tabular}{cccc}
			\hline
			Week & Date & Topics & Assignment/Exam \\
			\hline
			Week 1 & Jan 19 & Stress Intensity &  \\
			Week 2 & Jan 26 & Stress Intensity & \\
			Week 3 & Feb 2 & Plastic Zone & Homework 1 Due\\
			Week 4 & Feb 9 & Fracture Toughness & Homework 2 Due\\
			Week 5 & Feb 16 & Residual Strength & Homework 3 Due\\
			Week 6 & Feb 23 & Multiple Site Damage & Homework 4 Due\\
			Week 7 & Mar 1 & Mixed-Mode Fracture & Homework 5 Due\\
			Week 8 & Mar 8 & Exam Review & Exam 1 \\
			Week 9 & Mar 22 & Stress Based Fatigue & \\
			Week 10 & Mar 29 & Strain Based Fatigue & Homework 6 Due\\
			Week 11 & Apr 5 & Crack Growth & Homework 7 Due\\
			Week 12 & Apr 12 & Crack Retardation & Homework 8 Due\\
			Week 13 & Apr 19 & Exam Review & Exam 2\\
			Week 14 & Apr 26 & Damage Tolerance & \\
			Week 15 & May 3 & Special Topics & \\
			\hline
		\end{tabular}
\end{table}

\vspace*{.15in}
\noindent\textbf{Undergraduate vs. Graduate Credit:}
Undergraduate students enrolled in 700 level courses will receive undergraduate credit (not graduate credit) unless they have a previously approved senior rule application or dual/accelerated enrollment form on file in the Graduate School. Undergraduate credit earned in 700 level courses cannot later be counted toward a graduate degree.

\vspace*{.15in}
\noindent\textbf{Important Academic Dates:}
For the Spring 2016 Semester, classes begin January 19 and end May 5.
The last date to drop a class and receive a W (withdrawn) instead of an F (failed) is April 1.

\vspace*{.15in}
\noindent\textbf{Disabilities:}
If you have a physical, psychiatric/emotional, or learning disability that may impact on your ability to carry out assigned course work, I encourage you to contact the Office of Disability Services (DS).
The office is located in Grace Wilkie Annex, room 150, (316) 978-3309 (voice/tty) (316-854-3032 videophone). DS will review your concerns and determine, with you, what academic accommodations are necessary and appropriate for you. All information and documentation of your disability is confidential and will not be released by DS without your written permission.

\vspace*{.15in}
\noindent\textbf{Counseling \& Testing:}
The WSU Counseling \& Testing Center provides professional counseling services to students, faculty and staff; administers tests and offers test preparation workshops; and presents programs on topics promoting personal and professional growth. Services are low cost and confidential. They are located in room 320 of Grace Wilkie Hall, and their phone number is (316) 978-3440. The Counseling \& Testing Center is open on all days that the University is officially open. If you have a mental health emergency during the times that the Counseling \& Testing Center is not open, please call COMCARE Crisis Services at (316) 660-7500.

\vspace*{.15in}
\noindent\textbf{Diversity and Inclusive:}
Wichita State University is committed to being an inclusive campus that reflects the evolving diversity of society. To further this goal, WSU does not discriminate in its programs and activities on the basis of race, religion, color, national origin, gender, age, sexual orientation, gender identity, gender expression, marital status, political affiliation, status as a veteran, genetic information or disability. The following person has been designated to handle inquiries regarding nondiscrimination policies: Executive Director, Office of Equal Opportunity, Wichita State University, 1845 Fairmount, Wichita KS 67260-0138; telephone (316) 978-3186.

\vspace*{.15in}
\noindent\textbf{Intellectual Property:}
Wichita State University students are subject to Board of Regents and University policies (see \url{http://webs.wichita.edu/inaudit/ch9_10.htm}) regarding intellectual property rights. Any questions regarding these rights and any disputes that arise under these policies will be resolved by the President of the University, or the President’s designee, and such decision will constitute the final decision.

\vspace*{.15in}
\noindent\textbf{Shocker Alert System:}
Get the emergency information you need instantly and effortlessly! With the Shocker Alert System, we will contact you by email the moment there is an emergency or weather alert that affects the campus.  Sign up at \url{www.wichita.edu/alert}.

\vspace*{.15in}
\noindent\textbf{Title IX:}
Title IX of the Educational Amendments of 1972 prohibits discrimination based on sex in any educational institution that receives federal funding. Wichita State University does not tolerate sex discrimination of any kind including: sexual misconduct; sexual harassment; relationship/sexual violence and stalking.  These incidents may interfere with or limit an individual’s ability to benefit from or participate in the University’s educational programs or activities. Students are asked to immediately report incidents to the University Police Department, (316) 978- 3450 or the Title IX Coordinator (316) 978-5177. Students may also report incidents to an instructor, faculty or staff member, who are required by law to notify the Title IX Coordinator. If a student wishes to keep the information confidential, the student may speak with staff members of the Counseling and Testing Center (316) 978-3440 or Student Health Services (316)978-3620. For more information about Title IX, go to: \url{http://www.wichita.edu/thisis/home/?u=titleixf}

%%%%%% THE END 
\end{document} 