%Do not change 
\documentclass[12pt, oneside]{article}
\usepackage{amssymb,amsmath}
\usepackage[margin=1in]{geometry}
\usepackage{textpos}
\usepackage{float}
%\usepackage{color}
\usepackage{graphicx}
\usepackage{tikz}
\usetikzlibrary{positioning}
\usepackage{tikz-3dplot}
\usepackage{pgfopts}
\usepackage{wasysym}
\usepackage{structuralanalysis}
\usepackage{fancyhdr}
\usepackage[en-US]{datetime2}

\fancyhead[R]{\small Last Updated: \today\ at \DTMcurrenttime}
\pagestyle{fancy}

% You may add the packages you need here



\begin{document}
% Do not modify 


%Do not modify
\begin{textblock*}{4cm}(-1.7cm,-2.3cm)
\noindent {\scriptsize AE737 Spring 2016} 
\end{textblock*}

%Do not modify other than putting your name where stated
\begin{textblock*}{8cm}(12.5cm,-1cm)
\noindent {Name: } 
\end{textblock*}
%Do not modify other than typing your acknowledgement where stated
\begin{textblock*}{13.5cm}(-1.7cm,-1.8cm)
%\noindent \textit{\footnotesize Acknowledgement: Your acknowledgement for collaboration and other sources goes here. } 
\end{textblock*}

\vspace{1cm}

%Do not modify other than typing the homework number after #
\begin{center}
\textbf{\Large Homework 6}

\textbf{Due 29 Mar 2016}
\end{center}

%Rest should contain your solution for the homework. Feel free to improvise in ways that you believe make grading easier.
\begin{enumerate}

%calculate and plot residual strength for multiple materials
\begin{figure}[H]
	\item Plot the fatigue data given in hw6\_data.txt and, using the data, find $\sigma_f^\prime$ and $b$.
\end{figure}

\begin{figure}[H]
	\item Use the data and results from problem 1 to estimate the S-N curve for an identical specimen loaded with a mean stress of 30 ksi.
\end{figure}

\begin{figure}[H]
	\item The following state of constant amplitude fatigue stress is applied to an un-notched specimen of 2024-T4 aluminum
	\begin{align*}
	\sigma_x &= 27 \text{ ksi}\\
	\sigma_y &= 13 \text{ ksi}\\
	\tau_{xy} &= 8 \text{ ksi}
	\end{align*}
	Predict the number of cycles before failure.
\end{figure}

\end{enumerate}
\end{document}
