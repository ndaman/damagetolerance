%Do not change 
\documentclass[12pt, oneside]{article}
\usepackage{amssymb,amsmath}
\usepackage[margin=1in]{geometry}
\usepackage{textpos}
\usepackage{float}
%\usepackage{color}
\usepackage{graphicx}
\usepackage{tikz}
\usetikzlibrary{positioning}
\usepackage{tikz-3dplot}
\usepackage{pgfopts}
\usepackage{wasysym}
\usepackage{structuralanalysis}
\usepackage{fancyhdr}
\usepackage[en-US]{datetime2}

\fancyhead[R]{\small Last Updated: \today\ at \DTMcurrenttime}
\pagestyle{fancy}

% You may add the packages you need here



\begin{document}
% Do not modify 


%Do not modify
\begin{textblock*}{4cm}(-1.7cm,-2.3cm)
\noindent {\scriptsize AE737 Spring 2016} 
\end{textblock*}

%Do not modify other than putting your name where stated
\begin{textblock*}{8cm}(12.5cm,-1cm)
\noindent {Name: } 
\end{textblock*}
%Do not modify other than typing your acknowledgement where stated
\begin{textblock*}{13.5cm}(-1.7cm,-1.8cm)
%\noindent \textit{\footnotesize Acknowledgement: Your acknowledgement for collaboration and other sources goes here. } 
\end{textblock*}

\vspace{1cm}

%Do not modify other than typing the homework number after #
\begin{center}
\textbf{\Large Homework 8}

\textbf{Due 19 Apr 2016}
\end{center}

%Rest should contain your solution for the homework. Feel free to improvise in ways that you believe make grading easier.
\begin{enumerate}

\begin{figure}[H]
	\item A wide, center-cracked specimen with Paris law parameters $C=10^{-9}$ and $n=4$ has an initial crack length of $2a = 2 \text{ in}$. 
	The specimen is subjected to an $R=0$ cyclic stress such that $\sigma = 3.4a^{-1}$, where $a$ is the current crack length. 
	How many cycles will it take for the crack to reach $2a = 8 \text{ in}$?
\end{figure}

\begin{figure}[H]
	\item An edge-cracked specimen with Paris law parameters $C=10^{-9}$ and $n=4$ has an initial crack length of $a = 0.5 \text{ in}$. 
	The specimen is subjected to an $R=0$ cyclic stress such that $\sigma = 3.0 \text{ ksi}$.
	What will the crack length be after 50,000 cycles?
\end{figure}

\begin{figure}[H]
	\item While flicking the clip on his pen, Dr. Smith (with his eagle vision) notices a 0.01" edge-crack.
	Assume that Dr. Smith's flicks are generally about 3 in-lbs, with one strong, 5 in-lb flick for every 10 regular flicks.
	If the pen clip is 0.25" wide, 0.05" thick and made from 7075-T6 (with $K_c = 70 \text{ ksi}\sqrt{\text{in}}$), use the Boeing-Walker growth rate equation with $p=3.5$, $q=0.6$, $\mu=0.1$, and $m_T = 24$ to estimate the number of cycles remaining for Dr. Smith's pen.
\end{figure}

\begin{figure}[H]
	\item Consider a wide, center-cracked panel with an initial crack length of $2a = 0.5 \text{ in}$.
	Use the Boeing-Walker growth rate equation with $p=3.5$, $q=0.6$, $\mu=0.1$, and $m_T = 24$.
	Compare the expected crack growth rate for the expected load ($\sigma_{min} = 5 \text{ ksi}$ and $\sigma_{max} = 20 \text{ ksi}$) with a situation where an unexpected overload ($\sigma_{min} = 5 \text{ ksi}$ and $\sigma_{max} = 40 \text{ ksi}$) occurs after 5,000 cycles
	\begin{enumerate}
		\item Using the Wheeler retardation model with $m = 1.5$ and a plane stress plastic zone for $\sigma_{ys} = 68 \text{ ksi}$
		\item Using the Willenborg retardation model with $S_{OL} = 2.0$ and $K^{th} = 3 \text{ ksi}\sqrt{\text{in}}$.
		\item Using the closure retardation model with $C_{f0} = 0.3$
	\end{enumerate}
\end{figure}

\end{enumerate}
\end{document}
