%Do not change 
\documentclass[12pt, oneside]{article}
\usepackage{amssymb,amsmath}
\usepackage[margin=1in]{geometry}
\usepackage{textpos}
\usepackage{float}
%\usepackage{color}
\usepackage{graphicx}
\usepackage{tikz}
\usetikzlibrary{positioning}
\usepackage{tikz-3dplot}
\usepackage{pgfopts}
\usepackage{wasysym}
\usepackage{structuralanalysis}
\usepackage{fancyhdr}
\usepackage[en-US]{datetime2}

\fancyhead[R]{\small Last Updated: \today\ at \DTMcurrenttime}
\pagestyle{fancy}

% You may add the packages you need here



\begin{document}
% Do not modify 


%Do not modify
\begin{textblock*}{4cm}(-1.7cm,-2.3cm)
\noindent {\scriptsize AE737 Spring 2016} 
\end{textblock*}

%Do not modify other than putting your name where stated
\begin{textblock*}{8cm}(12.5cm,-1cm)
\noindent {Name: } 
\end{textblock*}
%Do not modify other than typing your acknowledgement where stated
\begin{textblock*}{13.5cm}(-1.7cm,-1.8cm)
%\noindent \textit{\footnotesize Acknowledgement: Your acknowledgement for collaboration and other sources goes here. } 
\end{textblock*}

\vspace{1cm}

%Do not modify other than typing the homework number after #
\begin{center}
\textbf{\Large Homework 7}

\textbf{Due 5 Apr 2016}
\end{center}

%Rest should contain your solution for the homework. Feel free to improvise in ways that you believe make grading easier.
\begin{enumerate}

%calculate and plot residual strength for multiple materials
\begin{figure}[H]
	\item For the plate shown below (Al 2024-T3) determine the number of cycles to failure for a fully reversed sinusoidal load of $P_a = 30$ k-lbs.
	\begin{enumerate}
		\item Find $K_t$
		\item Estimate $k_f$
		\item Find the cycles to failure
	\end{enumerate}
	\centering
	\begin{tikzpicture}
	\draw (-2,3) -- (2,3) -- (2,-3) -- (-2,-3) -- (-2,3);
	\draw (0,0) circle (0.5cm);
	\draw[->] (0,3) -- (0,4) node[above] {$P_a = 30$ k-lbs};
	\draw[->] (0,-3) -- (0,-4) node[below] {$P_a = 30$ k-lbs};
	\draw[->] (-0.5,-2) node at (0,-2) {7"} -- (-2,-2);
	\draw[->] (0.5,-2) -- (2,-2);
	\draw[->] (0.5,2) node[right] {.25"$\diameter$} -- (0,0.5); 
	\draw node at (4,0) {$t = 0.157 \text{in.}$};
	\end{tikzpicture}
	%\caption{Plate for Problem 3}
	\label{fig:problem3}
\end{figure}

%TODO give E in problem 2
\begin{figure}[H]
	\item Use the included data file, hw7.txt, which contains strain life experimental data, to calculate the material properties $\sigma_f^\prime$, $b$, $\epsilon_f^\prime$, and $c$
\end{figure}

\begin{figure}[H]
	\item For the above data, find the transition fatigue life, $N_t$
\end{figure}

\begin{figure}[H]
	\item Plot the effects of a mean stress of $\sigma_m = 20$ ksi on the above data, assuming $E = 10,300 \text{ ksi}$, $H^\prime = 142 \text{ ksi}$ and $n^\prime = 0.106$. Compare the Morrow approach, modified Morrow equation, and Smith, Watson, and Topper approach.
\end{figure}

\end{enumerate}
\end{document}
