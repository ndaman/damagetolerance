%Do not change 
\documentclass[12pt, oneside]{article}
\usepackage{amssymb,amsmath}
\usepackage[margin=1in]{geometry}
\usepackage{textpos}
\usepackage{float}
%\usepackage{color}
\usepackage{graphicx}
\usepackage{tikz}
\usetikzlibrary{positioning}
\usepackage{tikz-3dplot}
\usepackage{pgfopts}
\usepackage{wasysym}
\usepackage{structuralanalysis}
\usepackage{fancyhdr}
\usepackage[en-US]{datetime2}

\fancyhead[R]{\small Last Updated: \today\ at \DTMcurrenttime}
\pagestyle{fancy}

% You may add the packages you need here



\begin{document}
% Do not modify 


%Do not modify
\begin{textblock*}{4cm}(-1.7cm,-2.3cm)
\noindent {\scriptsize AE737 Spring 2016} 
\end{textblock*}

%Do not modify other than putting your name where stated
\begin{textblock*}{8cm}(12.5cm,-1cm)
\noindent {Name: } 
\end{textblock*}
%Do not modify other than typing your acknowledgement where stated
\begin{textblock*}{13.5cm}(-1.7cm,-1.8cm)
%\noindent \textit{\footnotesize Acknowledgement: Your acknowledgement for collaboration and other sources goes here. } 
\end{textblock*}

\vspace{1cm}

%Do not modify other than typing the homework number after #
\begin{center}
\textbf{\Large Homework 3}

\textbf{Due 16 Feb 2016}
\end{center}


%Rest should contain your solution for the homework. Feel free to improvise in ways that you believe make grading easier.
\begin{enumerate}

%from Grandt's text
\begin{figure}[H]
	\item A large steel sheet (0.1 in. thick) containing a center-crack with length $2a = 4 \text{in.}$ fractures under remote tensile stress of $\sigma = 30 \text{ksi}$. This material has a yield strength of 100 ksi. If you feel you cannot answer any of the following questions with the given data, explain why not.
	\begin{enumerate}
		\item What is the fracture toughness for this sheet of steel?
		\item What is $K_{IC}$ for this material?
		\item What remote stress, $\sigma$ will fracture an identical plate with a 9 in. crack?
		\item What is the failure stress for a 4-in. long center crack in a 1-in. thick plate?
		\item What is the fracture stress if the original 0.1 in. thick panel contains a 0.1 in. center crack?
	\end{enumerate}
\end{figure}

%Smith original problem
\begin{figure}[H]
	\item Consider a panel with an edge crack $a=2 \text{ in.}$. Predict the failure stress for the following conditions
	\begin{enumerate}
		\item The panel is Aluminum 2024-T351 (L-T direction), 1.5 in. thick, 8 in. wide, and $\sigma_{YS} = 50 \text{ ksi}$, at room temperature.
		\item Consider an identical panel, but with a crack in the transverse direction (T-L).
		\item The panel is Aluminum 7075-T6 (L-T direction), 0.5 in. thick, 8 in. wide, and $\sigma_{YS} = 72 \text{ ksi}$ at room temperature.
		\item Consider the same (7075-T6) panel, at -65$^\circ$F.
		\item The panel is Steel 15-5 PH, 1.0 in. thick, 8 in. wide, with $\sigma_{YS} = 140 \text{ ksi}$
	\end{enumerate}
\end{figure}

%Smith original, resistance curve (ASTM 561)
\begin{figure}[H]
	\item The data file HW3-3.txt contains load-displacement data (P vs. v) for an aluminum alloy. Assuming the specimen had a thickness of 0.05", a yield stress of $\sigma_{YS} = 48 \text{ ksi}$, a width of 50", Young's Modulus $E = 11 \text{Msi}$, Poisson's ratio of $\nu = 0.33$ and half-span displacement measurement points at $Y = 0.75$":
	\begin{enumerate}
		\item use the secant method to calculate the effective crack length throughout the test and plot the $K_R$ curve
		\item find $K_C$ using the tangent curve method
	\end{enumerate} 
\end{figure}

\end{enumerate}
\end{document}
