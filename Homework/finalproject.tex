%Do not change 
\documentclass[12pt, oneside]{article}
\usepackage{amssymb,amsmath}
\usepackage[margin=1in]{geometry}
\usepackage{textpos}
\usepackage{float}
%\usepackage{color}
\usepackage{graphicx}
\usepackage{tikz}
\usetikzlibrary{positioning}
\usepackage{tikz-3dplot}
\usepackage{pgfopts}
\usepackage{wasysym}
\usepackage{structuralanalysis}
\usepackage{fancyhdr}
\usepackage[en-US]{datetime2}

\fancyhead[R]{\small Last Updated: \today\ at \DTMcurrenttime}
\pagestyle{fancy}

% You may add the packages you need here



\begin{document}
% Do not modify 


%Do not modify
\begin{textblock*}{4cm}(-1.7cm,-2.3cm)
\noindent {\scriptsize AE737 Spring 2016} 
\end{textblock*}

%Do not modify other than putting your name where stated
\begin{textblock*}{8cm}(12.5cm,-1cm)
\noindent {Name: } 
\end{textblock*}
%Do not modify other than typing your acknowledgement where stated
\begin{textblock*}{13.5cm}(-1.7cm,-1.8cm)
%\noindent \textit{\footnotesize Acknowledgement: Your acknowledgement for collaboration and other sources goes here. } 
\end{textblock*}

\vspace{1cm}

%Do not modify other than typing the homework number after #
\begin{center}
\textbf{\Large Final Project}

\textbf{Due 10 May 2016}
\end{center}

Choose any real life object which undergoes cyclic loading (could be a car part, pen clip. aircraft wing, any object).
For this object, identify a location which is either likely to develop a crack due to cyclic loading, or is in some critical location should a crack occur there.
Using the methods learned in this course
\begin{itemize}
	\item estimate the stress intensity factor (you will likely want to consider a plot, or multiple plots to describe the possible conditions)
	\item estimate the residual strength once a crack has developed
	\item from some realistic loading cycle, estimate the fatigue life
	\item and crack propagation
	\item based on the above findings, propose (and justify) an appropriate inspection cycle
	\item suggest at least one design improvement that could make this object more damage tolerant
\end{itemize}

You will likely need to make simplifying assumptions and material substitutions at many, if not all, stages of this project.
Demonstrate your engineering and damage tolerance knowledge by justifying these assumptions and/or substitutions as concretely as possible.
(For example, if you make the assumption that your part is in plane stress, show how accurate this is by calculating $I$.)

Remember that this project is in place of a final exam.
While choosing a simple part or crack geometry may be appropriate, full credit will only be given when a thorough understanding of course material has been demonstrated.
As an example, if your chosen object is most likely to have a center crack in a very wide section, you should consider some extenuating circumstances, such as mixed-mode fracture, adjacent geometry (holes), or multiple-site damage.

Assume that the reader of this report has a basic understanding of fracture mechanics, and is generally trusting of your calculations.
You need not include calculation details, but you should cite the equations you use, explain the parameters entered into the calculation, and explain what the results mean.
Plots and figures are generally helpful in understanding the overall results.
Reports should be no longer than 20 pages.

\newpage
Projects will be graded on the following rubric
\begin{itemize}
	\item Project abstract - 5\% (due 31 March 2016)
	\item Stress intensity factor analysis - 10\%
	\item Residual strength analysis - 10\%
	\item Fatigue analysis (stress/strain based fatigue) - 10\%
	\item Fatigue crack propagation analysis (fracture mechanics based fatigue) - 10\%
	\item Inspection cycle analysis - 10\%
	\item Damage tolerant improvement - 20\%
	\item General presentation, organization, and grammar - 25\%
\end{itemize}
\end{document}
