%Do not change 
\documentclass[12pt, oneside]{article}
\usepackage{amssymb,amsmath}
\usepackage[margin=1in]{geometry}
\usepackage{textpos}
\usepackage{float}
%\usepackage{color}
\usepackage{graphicx}
\usepackage{tikz}
\usetikzlibrary{positioning}
\usepackage{tikz-3dplot}
\usepackage{pgfopts}
\usepackage{wasysym}
\usepackage{structuralanalysis}
\usepackage{fancyhdr}
\usepackage[en-US]{datetime2}

\fancyhead[R]{\small Last Updated: \today\ at \DTMcurrenttime}
\pagestyle{fancy}

% You may add the packages you need here



\begin{document}
% Do not modify 


%Do not modify
\begin{textblock*}{4cm}(-1.7cm,-2.3cm)
\noindent {\scriptsize AE737 Spring 2016} 
\end{textblock*}

%Do not modify other than putting your name where stated
\begin{textblock*}{8cm}(12.5cm,-1cm)
\noindent {Name: } 
\end{textblock*}
%Do not modify other than typing your acknowledgement where stated
\begin{textblock*}{13.5cm}(-1.7cm,-1.8cm)
%\noindent \textit{\footnotesize Acknowledgement: Your acknowledgement for collaboration and other sources goes here. } 
\end{textblock*}

\vspace{1cm}

%Do not modify other than typing the homework number after #
\begin{center}
\textbf{\Large Final Project Abstract}

\textbf{Due 31 March 2016}
\end{center}

Choose any real life object which undergoes cyclic loading (could be a car part, pen clip. aircraft wing, any object).
In 1-2 pages, give a brief overview of how this project will sufficiently satisfy the requirements for the final project.
Projects will be graded on the following rubric
\begin{itemize}
	\item Project abstract - 5\% (due 31 March 2016)
	\item Stress intensity factor analysis - 10\%
	\item Residual strength analysis - 10\%
	\item Fatigue crack growth analysis - 10\%
	\item Fatigue crack propagation analysis - 10\%
	\item Inspection cycle analysis - 10\%
	\item Damage tolerant improvement - 20\%
	\item General presentation, organization, and grammar - 25\%
\end{itemize}

The purpose of this abstract is to get you thinking about your final project before it is too late, and to ensure that you do not spend effort on a project which does not have features needed to satisfy this.
Take some time to think about your chosen object, some of the basic assumptions you will need to make, and if you will be able to demonstrate a thorough understanding of course material using it in your final project.
\end{document}
